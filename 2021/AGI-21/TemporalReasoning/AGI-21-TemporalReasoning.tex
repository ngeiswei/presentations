\documentclass[aspectratio=169]{beamer}

\usepackage{beamerthemesplit}
\usepackage{amsmath}
\usepackage{amsfonts}
\usepackage{amssymb}
\usepackage{cancel}
\usepackage{bussproofs}
%% \usepackage{tkz-graph}

\makeatletter
\newcommand{\reallytiny}{\@setfontsize{\srcsize}{2pt}{2pt}}
\makeatother

\mode<presentation>
{
  \usetheme{AnnArbor}
  \usecolortheme{crane}
}

\usepackage[english]{babel}
\usepackage[latin1]{inputenc}
\usepackage{times}
\usepackage[T1]{fontenc}

\title{Temporal Reasoning with OpenCog}

\author{Nil Geisweiller}

\institute[SingularityNET OpenCog Foundations]
{
  \begin{center}
    SingularityNET \& OpenCog Foundations\\
    \includegraphics[scale=0.32]{pictures/snet_oc.png}
  \end{center}
}

\date[AGI-21]

\begin{document}

\section {Introduction}

\begin{frame}
  \maketitle
\end{frame}

\begin{frame}
  \frametitle{Why Temporal Reasoning?}

  \begin{enumerate}
    % We all know why, we live in a temporal universe, and in this
    % universe there are lags between causes and effects and we need
    % to be able to reasoning about these.
  \item Lag between cause and effect
    % And that gonna include your own thinking.  When your doing
    % meta-reasoning you need to be able to control your reasoning, at
    % some point to cut your never ending thinking loop with "Wait I
    % don't have time to think anymore", stop thinking and start
    % acting, but to think that you don't have time to think anymore,
    % you need to be able to think about time.
  \item Meta-reasoning: {\large Think about} {\small think about}
    {\footnotesize think about} {\tiny think about ...}
  %% \item Keep thinking or act?
  \end{enumerate}
\end{frame}

\begin{frame}[fragile]
  \frametitle{PLN Recall}

  %% Given a list of predicates
\texttt{P}, \texttt{Q}, $\hdots$: $Atom^n \rightarrow \{True, False\}$

  \begin{columns}
    \column{1in}

\begin{semiverbatim}
  And <TV>
    P
    Q
\end{semiverbatim}

    \column{0.5in}
    $$\equiv$$

    \column{1in}
    $$\mathcal{P}(P,Q) \approx TV.strength$$

  \end{columns}

  \begin{columns}
    \column{1in}

\begin{semiverbatim}
  Not <TV>
    P
\end{semiverbatim}

    \column{0.5in}
    $$\equiv$$

    \column{1in}
    $$\mathcal{P}(P) \approx 1 - TV.strength$$

  \end{columns}

  \begin{columns}
    \column{1in}

\begin{semiverbatim}
  Implication <TV>
    P
    Q
\end{semiverbatim}

    \column{0.5in}
    $$\equiv$$

    \column{1in}
    $$\mathcal{P}(Q|P) \approx TV.strength$$

  \end{columns}

\end{frame}

\begin{frame}[fragile]

  \frametitle{PLN rules: Implication Direct Evaluation}
\begin{semiverbatim}
Evaluation
  P
  Ei
...
Evaluation
  Q
  Ei
|-
Implication <TV>
  P
  Q
\end{semiverbatim}

  %% \begin{prooftree}
  %%   \AxiomC{$\texttt{Evaluation <TV> P Ei}$}
  %%   \AxiomC{$\texttt{Evaluation <TV> Q Ei}$}
  %%   \BinaryInfC{$\texttt{Implication <TV> P Q}$}
  %% \end{prooftree}


$$TV.strength = \frac{\sum_x f_\wedge(P(x).strength, Q(x).strength)}{\sum_x P(x).strength}$$
\end{frame}

\begin{frame}[fragile]

  \frametitle{PLN rules: Deduction}
\begin{semiverbatim}
Implication
  P
  Q
Implication
  Q
  R
|-
Implication <TV>
  P
  R
\end{semiverbatim}
$$TV.strength = \mathcal{P}(R|Q,P)\times\mathcal{P}(Q|P) + \mathcal{P}(R|¬Q,P)\times\mathcal{P}(¬Q|P)$$
\end{frame}

\section {Temporal Reasoning with PLN}

\begin{frame}
  \frametitle{Temporal Predicate}
  $$P: Atom^n \times T \rightarrow \{True, False\}$$
  $$P: \texttt{\_-\_--\_}$$
  $$Q: $$
\end{frame}

\begin{frame}[fragile]
  \frametitle{SequentialAnd}

  \begin{columns}
    \column{1in}
\begin{semiverbatim}
BackSequentialAnd <TV>
  L
  P
  Q
\end{semiverbatim}

    \column{0.5in}
    $$\equiv$$

    \column{1in}
\begin{semiverbatim}
And <TV>
  Lag
    L
    P
  Q
\end{semiverbatim}

  \end{columns}

    \begin{columns}
    \column{1in}
\begin{semiverbatim}
ForeSequentialAnd <TV>
  L
  P
  Q
\end{semiverbatim}

    \column{0.5in}
    $$\equiv$$

    \column{1in}
\begin{semiverbatim}
And <TV>
  P
  Lead
    L
    Q
\end{semiverbatim}

  \end{columns}

\end{frame}


\begin{frame}[fragile]
  \frametitle{PredictiveImplication}

  \begin{columns}
    \column{1in}
\begin{semiverbatim}
BackPredictiveImplication <TV>
  L
  P
  Q
\end{semiverbatim}

    \column{0.5in}
    $$\equiv$$

    \column{1in}
\begin{semiverbatim}
Implication <TV>
  Lag
    L
    P
  Q
\end{semiverbatim}

  \end{columns}

    \begin{columns}
    \column{1in}
\begin{semiverbatim}
ForePredictiveImplication <TV>
  L
  P
  Q
\end{semiverbatim}

    \column{0.5in}
    $$\equiv$$

    \column{1in}
\begin{semiverbatim}
Implication <TV>
  P
  Lead
    L
    Q
\end{semiverbatim}

  \end{columns}

\end{frame}

\begin{frame}[fragile]
  \frametitle{PredictiveImplication}

  \begin{columns}
    \column{1in}
\begin{semiverbatim}
BackPredictiveImplication <TV>
  L
  P
  Q
\end{semiverbatim}

    \column{0.5in}
    $$\equiv$$

    \column{1in}
\begin{semiverbatim}
Implication <TV>
  Lag
    L
    P
  Q
\end{semiverbatim}

  \end{columns}

    \begin{columns}
    \column{1in}
\begin{semiverbatim}
ForePredictiveImplication <TV>
  L
  P
  Q
\end{semiverbatim}

    \column{0.5in}
    $$\equiv$$

    \column{1in}
\begin{semiverbatim}
Implication <TV>
  P
  ForeSequentialAnd
    L
    P
    Q
\end{semiverbatim}

  \end{columns}

\end{frame}

\end{document}
