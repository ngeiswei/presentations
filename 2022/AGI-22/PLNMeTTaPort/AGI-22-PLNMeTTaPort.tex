\documentclass[aspectratio=169]{beamer}

\usepackage{beamerthemesplit}
\usepackage{amsmath}
\usepackage{amsfonts}
\usepackage{amssymb}
\usepackage{cancel}
\usepackage{bussproofs}
%% \usepackage{tkz-graph}

% For ⩘ and ⩗ (requires the LuaLaTeX engine)
\usepackage{unicode-math}
\setmathfont{Stix Two Math}

% Commands for Atomese code
\newcommand{\SP}{\;\;\;}
\newcommand{\TTrue}{\textit{True}}
\newcommand{\TFalse}{\textit{False}}
\newcommand{\TAtom}{\textit{Atom}}
\newcommand{\TTime}{\textit{Time}}
\newcommand{\TEval}{\textit{Evaluation}}
\newcommand{\TList}{\textit{List}}
\newcommand{\TLamb}{\textit{Lambda}}
\newcommand{\TExec}{\textit{Execution}}
\newcommand{\TAtTime}{\textit{AtTime}}
\newcommand{\TAnd}{\textit{And}}
\newcommand{\TOr}{\textit{Or}}
\newcommand{\TNot}{\textit{Not}}
\newcommand{\TImpl}{\textit{Implication}}
\newcommand{\TPredImpl}{\textit{PredictiveImplication}}
\newcommand{\TSeqAnd}{\textit{SequentialAnd}}
\newcommand{\TSeqOr}{\textit{SequentialOr}}
\newcommand{\TBSeqAnd}{\textit{BackSequentialAnd}}
\newcommand{\TFSeqAnd}{\textit{ForeSequentialAnd}}
\newcommand{\TLag}{\textit{Lag}}
\newcommand{\TLead}{\textit{Lead}}
\newcommand{\TTV}{\textit{tv}}
\newcommand{\TTVPi}{\textit{TV}_i^P}
\newcommand{\TTVQi}{\textit{TV}_i^Q}
\newcommand{\TTVP}{\textit{TV}^P}
\newcommand{\TTVQ}{\textit{TV}^Q}
\newcommand{\TTVR}{\textit{TV}^R}
\newcommand{\TTVPQ}{\textit{TV}^{PQ}}
\newcommand{\TTVQR}{\textit{TV}^{QR}}
\newcommand{\TBTV}{\langle \TTV \rangle}
\newcommand{\TBTVPi}{\langle \TTVPi \rangle}
\newcommand{\TBTVQi}{\langle \TTVQi \rangle}
\newcommand{\TBTVP}{\langle \TTVP \rangle}
\newcommand{\TBTVQ}{\langle \TTVQ \rangle}
\newcommand{\TBTVR}{\langle \TTVR \rangle}
\newcommand{\TBTVPQ}{\langle \TTVPQ \rangle}
\newcommand{\TBTVQR}{\langle \TTVQR \rangle}
\newcommand{\Tstrength}{\textit s}
\newcommand{\Tconf}{\textit c}

% Commands for symbolic mathematical notations
\newcommand{\prob}[1]{\mathcal{Pr}\left(#1\right)}
\newcommand{\mean}{\textit{mean}}
\newcommand{\sat}[1]{\mathcal{S}(#1)}
\newcommand{\limp}{\rightarrow}
\newcommand{\lpreimp}[1]{\leadsto^{#1}}
\newcommand{\lseqor}[1]{\bigslopedvee^{#1}}
\newcommand{\lseqand}[1]{\bigslopedwedge^{#1}}
\newcommand{\ldo}[1]{\widehat{#1}}
\newcommand{\llag}[2]{\overrightarrow{#1}^{#2}}
\newcommand{\llead}[2]{\overleftarrow{#1}^{#2}}

\makeatletter
\newcommand{\reallytiny}{\@setfontsize{\srcsize}{2pt}{2pt}}
\makeatother

\mode<presentation>
{
  \usetheme{AnnArbor}
  \usecolortheme{crane}
}

\usepackage[english]{babel}
%% \usepackage[latin1]{inputenc}
\usepackage{times}
\usepackage[T1]{fontenc}

\title{Towards porting PLN to MeTTa}

\author{Nil Geisweiller, Hedra Yusuf}

\institute[SingularityNET OpenCog Foundations]
{
  \begin{center}
    AGI-22 Workshop\\
    \includegraphics[scale=0.32]{pictures/snet_oc.png}
  \end{center}
}

\date[AGI-22]

\begin{document}

\section{Introduction}

\begin{frame}
  \maketitle
\end{frame}

%%%%%%%%%%%%%%%%%%%%%%%%%%%%%%%%%%%%%%%%%%%%%%%%%%%%%%%%%%%%%%%%%%%%%%%%%%%
%% This presentation is gonna be all about porting PLN to MeTTa.  You    %%
%% might see that the title has been somewhat simplified, that's because %%
%% we didn't have enough to really touch on the temporal aspect of PLN.  %%
%%%%%%%%%%%%%%%%%%%%%%%%%%%%%%%%%%%%%%%%%%%%%%%%%%%%%%%%%%%%%%%%%%%%%%%%%%%

\section{PLN Recall}

\begin{frame}[fragile]

%%%%%%%%%%%%%%%%%%%%%%%%%%%%%%%%%%%%%%%%%%%%%%%%%%%%%%%%%%%%%%%%%%%%%%%%%
%% So let me start by giving a brief recall of what is PLN, and as I   %%
%% do so I will also introduce some notations that will be used in the %%
%% PLN MeTTa port.                                                     %%
%%%%%%%%%%%%%%%%%%%%%%%%%%%%%%%%%%%%%%%%%%%%%%%%%%%%%%%%%%%%%%%%%%%%%%%%%

%%%%%%%%%%%%%%%%%%%%%%%%%%%%%%%%%%%%%%%%%%%%%%%%%%%%%%%%%%%%%%%%%%%%%%%
%% I'm only going to recall the portion of PLN that's relevant here, %%
%% not the whole logic.                                              %%
%%%%%%%%%%%%%%%%%%%%%%%%%%%%%%%%%%%%%%%%%%%%%%%%%%%%%%%%%%%%%%%%%%%%%%%

  \frametitle{PLN Recall}

%%%%%%%%%%%%%%%%%%%%%%%%%%%%%%%%
%% Given a list of predicates %%
%%%%%%%%%%%%%%%%%%%%%%%%%%%%%%%%

$P, Q, \hdots: \textit{Atom}^n \rightarrow \{\top, \bot\}$
{\tiny \alert{(possibly fuzzy)}}

\renewcommand{\arraystretch}{1.5}

{\small
$$
\begin{array}{|c|c|c|}
  \hline
  \text{Atomese} & \text{MeTTa} & \text{Math} \\
  \hline
  (P\ \TTV) & (\measeq P\ \TTV) & \prob{\sat{P}} \approx \TTV.\mean \\
  (\TNot\ \TTV\ P) & (\measeq (\lnot\ P)\ \TTV) & \prob{\overline{\sat{P}}}
  \approx \TTV.\mean \\
  (\TOr\ \TTV\ P\ Q) & (\measeq (\lor\ P\ Q)\ \TTV) & \prob{\sat{P}
  \cup \sat{Q}} \approx \TTV.\mean \\
  (\TAnd\ \TTV\ P\ Q) & (\measeq (\land\ P\ Q)\ \TTV) & \prob{\sat{P}
  \cap \sat{Q}} \approx \TTV.\mean \\
  (\TImpl\ \TTV\ P\ Q) & (\measeq (P \limp Q)\ \TTV) &
  \prob{\sat{Q}|\sat{P}} \approx \TTV.\mean \\
  (\TEval\ \TTV\ P\ (\TList\ X_1\ \dots\ X_n)) & (\measeq (P\ X_1\ \dots\ X_n)\
  \TTV) & \prob{P(X_1, \dots, X_n)=\top} \approx \TTV.\mean \\
  \hline
\end{array}
$$
}
\renewcommand{\arraystretch}{1}

\end{frame}

%%%%%%%%%%%%%%%%%%%%%%%%%%%%%%%%%%%%%%%%%%%%%%%%%%%%%%%%%%%%%%%%%%
%% Here we are going to try to port to MeTTa 2 inferences rules %%
%%%%%%%%%%%%%%%%%%%%%%%%%%%%%%%%%%%%%%%%%%%%%%%%%%%%%%%%%%%%%%%%%%

\begin{frame}[fragile]

%%%%%%%%%%%%%%%%%%%%%%%%%%%%%%%%%%%%%%%%%%%%%%%%%%%%%%%%%%%%%%%%%%%%%%%%
%% So it's pretty much in MeTTa but I used infix notations to make it %%
%% more readable.                                                     %%
%%%%%%%%%%%%%%%%%%%%%%%%%%%%%%%%%%%%%%%%%%%%%%%%%%%%%%%%%%%%%%%%%%%%%%%%

%% \frametitle{PLN rules: Full Deduction}

%% \begin{prooftree}
%%   \AxiomC{$P\limp Q \measeq \TTV$}
%%   \AxiomC{$Q\limp R \measeq \TTV$}
%%   \AxiomC{$\dots$}
%%   \TrinaryInfC{$P\limp R \measeq \TTV$}
%% \end{prooftree}

%% $\TTV.\mean = \prob{\sat{R}|\sat{Q}\cap\sat{P}} \times
%% \prob{\sat{Q}|\sat{P}} + \prob{\sat{R}|\overline{\sat{Q}}\cap\sat{P}}
%% \times \prob{\overline{\sat{Q}}|\sat{P}}$
%% \end{frame}

  \frametitle{PLN rules: Deduction}

  \begin{prooftree}
    \AxiomC{$P\limp Q \measeq \TTV_{PQ}$}
    \AxiomC{$Q\limp R \measeq \TTV_{QR}$}
    \AxiomC{$P \measeq \TTV_P$}
    \AxiomC{$Q \measeq \TTV_Q$}
    \AxiomC{$R \measeq \TTV_R$}
    \RightLabel{(DED)}
    \QuinaryInfC{$P\limp R \measeq \TTV$}
  \end{prooftree}

  {\small \alert{where}
    $$\TTV.\mean = \TTV_{PQ}.\mean \times \TTV_{QR}.\mean
    + \frac{(1 - \TTV_{PQ}.\mean) \times (\TTV_R.\mean - \TTV_Q.\mean \times \TTV_{QR}.\mean)}{1 -
      \TTV_Q.\mean}$$
  }
\end{frame}

\begin{frame}[fragile]

  \frametitle{PLN rules: Implication Direct Introduction}

  \begin{prooftree}
    \AxiomC{$(P\ a_1) \measeq \TTV_{Pa_1}$}
    \AxiomC{$(Q\ a_1) \measeq \TTV_{Qa_1}$}
    \AxiomC{$\dots$}
    \AxiomC{$(P\ a_n) \measeq \TTV_{Pa_n}$}
    \AxiomC{$(Q\ a_n) \measeq \TTV_{Qa_n}$}
    \RightLabel{(IDI)}
    \QuinaryInfC{$P\limp Q \measeq \TTV$}
  \end{prooftree}

  {\small \alert{where}
    $$\TTV.\mean = \frac{\sum_x \TTV_{Px}.\mean \times \TTV_{Qx}.\mean}{\sum_x
      \TTV_{Px}.\mean}$$
    }
\end{frame}

\section {Temporal Reasoning}

\begin{frame}
  \frametitle{Temporal Deduction $\mapsto$ Deduction}
  {\small
    \begin{prooftree}
      \AxiomC{$P \leadsto^{T_1} Q$}
      \RightLabel{(PI2I)}
      \UnaryInfC{$P \rightarrow \overleftarrow{Q}^{T_1}$}
      \AxiomC{$Q \leadsto^{T_2} R$}
      \RightLabel{(PI2I)}
      \UnaryInfC{$Q \rightarrow \overleftarrow{R}^{T_2}$}
      \RightLabel{(TS)}
      \UnaryInfC{$\overleftarrow{Q}^{T_1} \rightarrow \overleftarrow{R}^{T_1+T_2}$}
      \AxiomC{$P$}
      \AxiomC{$Q$}
      \RightLabel{(TS)}
      \UnaryInfC{$\overleftarrow{Q}^{T_1}$}
      \AxiomC{$R$}
      \RightLabel{(TS)}
      \UnaryInfC{$\overleftarrow{R}^{T_1+T_2}$}
      \RightLabel{(DED)}
      \QuinaryInfC{$P \rightarrow \overleftarrow{R}^{T_1+T_2}$}
      \RightLabel{(I2PI)}
      \UnaryInfC{$P \leadsto^{T_1+T_2} R$}
    \end{prooftree}}

  \begin{itemize}
  \item DED: Deduction
  \item TS: Temporal Shift
  \item PI2I: PredictiveImplication to Implication
  \item I2PI: Implication to PredictiveImplication
  \end{itemize}
\end{frame}

\section{Procedural Reasoning}

\begin{frame}
  \frametitle{Temporal Procedural Deduction $\mapsto$ Deduction}
  {\tiny
    \begin{prooftree}
      \AxiomC{$P \wedge \widehat{A} \leadsto^{T_1} Q$}
      \RightLabel{(PI2I)}
      \UnaryInfC{$P \wedge \widehat{A} \rightarrow
        \overleftarrow{Q}^{T_1}$}
      \AxiomC{$\widehat{B}$}
      \RightLabel{(TS)}
      \UnaryInfC{$\overleftarrow{\widehat{B}}^{T_1}$}
      \RightLabel{(CI)}
      \BinaryInfC{$P \wedge \widehat{A} \wedge
        \overleftarrow{\widehat{B}}^{T_1} \rightarrow
        \overleftarrow{Q}^{T_1} \wedge \overleftarrow{\widehat{B}}^{T_1}$}
      \AxiomC{$Q \wedge \widehat{B} \leadsto^{T_2} R$}
      \RightLabel{(PI2I)}
      \UnaryInfC{$Q \wedge \widehat{B} \rightarrow \overleftarrow{R}^{T_2}$}
      \RightLabel{(TS)}
      \UnaryInfC{$\overleftarrow{Q}^{T_1} \wedge
        \overleftarrow{\widehat{B}}^{T_1} \rightarrow
        \overleftarrow{R}^{T_1+T_2}$}
      %% \AxiomC{$P \wedge \widehat{A}$}
      %% \AxiomC{$\overleftarrow{\widehat{B}}^{T_1}$}
      %% \RightLabel{(CI)}
      \AxiomC{$P \wedge \widehat{A} \wedge \overleftarrow{\widehat{B}}^{T_1}$}
      \AxiomC{$Q \wedge \widehat{B}$}
      \RightLabel{(TS)}
      \UnaryInfC{$\overleftarrow{Q}^{T_1} \wedge \overleftarrow{\widehat{B}}^{T_1}$}
      \AxiomC{$R$}
      \RightLabel{(TS)}
      \UnaryInfC{$\overleftarrow{R}^{T_1+T_2}$}
      \RightLabel{(DED)}
      \QuinaryInfC{$P \wedge \widehat{A} \wedge \overleftarrow{\widehat{B}}^{T_1} \rightarrow \overleftarrow{R}^{T_1+T_2}$}
      \RightLabel{(I2PI)}
      \UnaryInfC{$\left( (P \wedge \widehat{A}) \lseqand{T_1}
        \widehat{B} \right) \leadsto^{T_1+T_2} R$}
    \end{prooftree}}

  \begin{itemize}
  \item CI: Conjunction Introduction
  \item TS: Temporal Shift
  \item DED: Deduction
  \item PI2I: PredictiveImplication to Implication
  \item I2PI: Implication to PredictiveImplication
  \end{itemize}

\end{frame}

\section{Conclusion}

\begin{frame}
  \frametitle{Conclusion}
  \begin{center} Demo Time \end{center}
\end{frame}
\end{document}
