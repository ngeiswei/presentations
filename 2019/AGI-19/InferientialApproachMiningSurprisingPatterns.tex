\documentclass{beamer}

\usepackage{beamerthemesplit}
\usepackage{amsmath}
\usepackage{amsfonts}
\usepackage{amssymb}
\usepackage{qtree}
\usepackage{cancel}
\usepackage{tkz-graph}
%\usepackage[pdftex]{graphicx}

\makeatletter
\newcommand{\reallytiny}{\@setfontsize{\srcsize}{2pt}{2pt}}
\makeatother

\mode<presentation>
{
  \usetheme{metropolis}
  % or ...

  %\setbeamercovered{transparent}
  % or whatever (possibly just delete it)
}

\usepackage[english]{babel}
% or whatever

\usepackage[latin1]{inputenc}
% or whatever

\usepackage{times}
\usepackage[T1]{fontenc}

\title{Inferential Approach to Mining Surprising Patterns in
  Hypergraphs}

\author{Nil Geisweiller, Ben Goertzel}

\institute[SingularityNET OpenCog Foundations] % (optional, but mostly needed)
{
  \begin{center}
    \includegraphics[scale=0.5]{images/snet_oc.png}
  \end{center}
}

\date[AGI-19] % (optional, should be abbreviation of conference name)
{AGI-19, Shenzhen}

\AtBeginSection[]
{
  \begin{frame}<beamer>{Outline}
    \tableofcontents[currentsection,currentsection]
  \end{frame}
}

\AtBeginSubsection[]
{
  \begin{frame}<beamer>{Outline}
    \tableofcontents[currentsection,currentsubsection]
  \end{frame}
}

%\newcommand{\AND}{\textit{AND}}
%\newcommand{\OR}{\textit{OR}}
%\newcommand{\NOT}{\textit{NOT}}
\newcommand{\AND}{\land}
\newcommand{\OR}{\lor}
\newcommand{\NOT}{\lnot}

\begin{document}

\frame
{
  \maketitle

%%%%%%%%%%%%
%% Speech %%
%%%%%%%%%%%%

%% This work starts with the desire to reframe learning as an explicit
%% form of reasoning.

%% We claim that by doing that, reframing learning as reasoning you
%% gain more transparency and as a result you make it much easier to
%% enable meta-learning at all levels.

%% Here in particular we are gonna focus on mining suprising patterns.

%% All the work I'm gonna present here is implemented with the OpenCog
%% framework, and it is sponsored by SingularityNET.

%%%%%%%%%%%%%
%% ~Speech %%
%%%%%%%%%%%%%
}

%% \section[Outline]{}
%% \frame{\tableofcontents}

\section{Introduction}

\begin{frame}
  \frametitle{Reframing \alert{learning as reasoning}}

%%%%%%%%%%%%
%% Speech %%
%%%%%%%%%%%%

%% OK, so what does reframing learning as reasoning mean?

%% What it means is, just think of any learning task, such as learning
%% how to recognize faces, which is translated as a task of proving a
%% certain theorem in a certain theory.

%%%%%%%%%%%%%
%% ~Speech %%
%%%%%%%%%%%%%
  
\end{frame}

\begin{frame}
  Reframe \alert{mining surprising patterns} as \alert{reasoning}
\end{frame}

\section{OpenCog recall}

\

\end{document}
