\documentclass[aspectratio=169]{beamer}

\usepackage{beamerthemesplit}
\usepackage{amsmath}
\usepackage{amsfonts}
\usepackage{amssymb}
\usepackage{cancel}
\usepackage{bussproofs}
%% \usepackage{tkz-graph}

\makeatletter
\newcommand{\reallytiny}{\@setfontsize{\srcsize}{2pt}{2pt}}
\makeatother

\mode<presentation>
{
  \usetheme{AnnArbor}
  \usecolortheme{crane}
}

\usepackage[english]{babel}
\usepackage[latin1]{inputenc}
\usepackage{times}
\usepackage[T1]{fontenc}

\title{AI-DSL for Autonomous Interoperability}

\author{Nil Geisweiller}

\institute[SingularityNET OpenCog Foundations]
{
  \begin{center}
    SingularityNET \& OpenCog Foundations\\
    \includegraphics[scale=0.32]{pics/snet_oc.png}
  \end{center}
}
          
\date[AI-DSL]

\begin{document}

\begin{frame}
  \maketitle
\end{frame}

\begin{frame}
  Instruction Path Length as approximation of Computing Time

  I suppose a good start is to consider monetary cost only, as it is
  entirely up to the agent. The agent doesn't need to be good at
  approximating its own cost as long as it abides by the
  contract. This allows us to initially let aside the computational
  cost. However it still requires the other parties to have an
  estimate of the quality of results relative to monetary cost.

  Another aspect is timing, so the agent must provide the cost of a
  service + the time it will take, and the serve (using CPUs or GPUs,
  etc) should be left to the agent, the user doesn't need to know
  about it.


  (agi : Nat) -> (expiry : Nat) -> A agi expiry -> B agi expiry
\end{frame}

\end{document}
