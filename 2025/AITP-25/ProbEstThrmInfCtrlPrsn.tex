\documentclass[aspectratio=169]{beamer}

\usepackage{beamerthemesplit}
\usepackage{amsmath}
\usepackage{amsfonts}
\usepackage{amssymb}
\usepackage{cancel}
\usepackage{bussproofs}
\usepackage{graphicx}
\usepackage{minted}
\usepackage{unicode-math}
\setmathfont{Stix Two Math}
\usepackage[pdf]{graphviz}

% For highlighting MeTTa code
\usepackage{listings}
\usepackage{color}
\definecolor{mygreen}{rgb}{0,0.6,0}
\definecolor{mymauve}{rgb}{0.58,0,0.82}
\definecolor{dgreen}{rgb}{0,0.5,0}
\definecolor{dblue}{rgb}{0,0,0.5}
\definecolor{dred}{rgb}{0.5,0,0}
\lstset{ %
  backgroundcolor=\color{white},   % choose the background color
  basicstyle=\tiny,                % size of fonts used for the code
  breaklines=true,                 % automatic line breaking only at whitespace
  captionpos=b,                    % sets the caption-position to bottom
  commentstyle=\color{mygreen},    % comment style
  escapeinside={\%*}{*)},          % if you want to add LaTeX within your code
  keywordstyle=\color{blue},       % keyword style
  stringstyle=\color{mymauve},     % string literal style
}

\makeatletter
\newcommand{\tinier}{\@setfontsize{\srcsize}{3.5pt}{3.5pt}}
\makeatother

\makeatletter
\newcommand{\tiniest}{\@setfontsize{\srcsize}{2pt}{2pt}}
\makeatother

\mode<presentation>
{
  \usetheme{AnnArbor}
  \usecolortheme{crane}
}

\usepackage[english]{babel}
%% \usepackage[latin1]{inputenc}
\usepackage{times}
\usepackage[T1]{fontenc}

\newcommand{\limp}{\Rightarrow}
\newcommand{\TheoryT}{\texttt{Theory}}
\newcommand{\ProofT}{\texttt{Proof}}
\newcommand{\PropositionT}{\texttt{Proposition}}
\newcommand{\BoolT}{\texttt{Bool}}
\newcommand{\True}{\texttt{True}}
\newcommand{\False}{\texttt{False}}
\newcommand{\axone}{\texttt{ax-1}}
\newcommand{\axtwo}{\texttt{ax-2}}
\newcommand{\axthree}{\texttt{ax-3}}
\newcommand{\axmp}{\texttt{ax-mp}}
\newcommand{\STV}[2]{<\!#1, #2\!>}

\title{Estimating the Probability of a Conjecture to be a Theorem in PLN for Inference Control}

\author{Nil Geisweiller}

\institute[SingularityNET]
{
  \begin{center}
    \includegraphics[scale=0.2]{figs/snet-logo.png}\\[1cm]
    Artificial Intelligence and Theorem Proving 2025 (AITP-25)
  \end{center}
}

\date[AITP-25]

\begin{document}

\lstset{language=Lisp}

\begin{frame}
  \maketitle
\end{frame}

\begin{frame}
  Ternary predicate relating theories, proofs and propositions:
  $$\Theta : {\color{violet}\TheoryT} \times {\color{blue}\ProofT} \times {\color{red}\PropositionT} \to \BoolT$$

  \begin{itemize}
  \item {\color{violet}$\TheoryT$}: \emph{Typing relationships} encoding axioms and
    inference rules.
    {\color{violet}$$\{\texttt{Z}:\texttt{Nat},\ \texttt{S}:\texttt{Nat} \texttt{->} \texttt{Nat}\}$$}
  \item {\color{blue}$\ProofT$}: \emph{Inhabitant} of a type.
    {\color{blue}$$\texttt{(S (S (S Z)))}$$}
  \item {\color{red}$\PropositionT$}: \emph{Type}.
    {\color{red}$$\texttt{Nat}$$}
  \end{itemize}
  $$\Theta({\color{violet}\{\texttt{Z}:\texttt{Nat},\ \texttt{S}:\texttt{Nat}
  \texttt{->} \texttt{Nat}\}}, {\color{blue}\texttt{(S (S (S Z)))}}, {\color{red}\texttt{Nat}})
  = \True$$
\end{frame}

\begin{frame}[fragile]
  \frametitle{Example (Propositional Calculus):}
  $
  \begin{array}{lll}
  \Theta & ( & {\color{violet}\{\ } \\
  & & \ \ \ {\color{violet} \axone \ :\ (\phi\ \to\ (\psi\ \to\ \phi)),} \\
  & & \ \ \ {\color{violet}\axtwo \ :\ ((\phi\ \to\ (\psi\ \to\ \chi))\ \to\ ((\phi\ \to\ \psi)\ \to\ (\phi\ \to\ \chi))),} \\
  & & \ \ \ {\color{violet}\axthree \ :\ (((\neg\ \phi)\ \to\ (\neg\ \psi))\ \to\ (\psi\ \to\ \phi)),} \\
  & & \ \ \ {\color{violet}\axmp \ :\ \phi\ \texttt{->}\ (\phi\ \to\ \psi)\ \texttt{->}\ \psi}
  \\
  & & {\color{violet} \} }, \\
  & &
  {\color{blue}
    (\lambda\ (\texttt{mp2.1}\ :\ \phi)\ (\texttt{mp2.3}\ :\ (\phi\ \to\ (\psi\ \to\ \chi)))\ (\axmp\ \texttt{mp2.1}\ \texttt{mp2.3}))
    },
  \\
  & &
  {\color{red}
    \phi\ \texttt{->}\ (\phi\ \to\ (\psi\ \to\ \chi))\ \texttt{->}\ (\psi\ \to\ \chi) }
  \\
  & ) & = \ \True
  \end{array}$
\end{frame}

\begin{frame}[fragile]
  \frametitle{Example (Propositional Calculus):}
  $
  \begin{array}{lll}
  \Theta & ( & {\color{violet}\{\ } \\
  & & \ \ \ {\color{violet}\axmp \ :\ \phi\ \texttt{->}\ (\phi\ \to\ \psi)\ \texttt{->}\ \psi}
  \\
  & & {\color{violet} \} }, \\
  & &
  {\color{blue}
    (\lambda\ (\texttt{mp2.1}\ :\ \phi)\ (\texttt{mp2.3}\ :\ (\phi\ \to\ (\psi\ \to\ \chi)))\ (\axmp\ \texttt{mp2.1}\ \texttt{mp2.3}))
    },
  \\
  & &
  {\color{red}
    \phi\ \texttt{->}\ (\phi\ \to\ (\psi\ \to\ \chi))\ \texttt{->}\ (\psi\ \to\ \chi) }
  \\
  & ) & = \ \True
  \end{array}$
\end{frame}

\begin{frame}[fragile]
  \frametitle{Example (Propositional Calculus):}
  $
  \begin{array}{lll}
  \Theta & ( & {\color{violet}\{\ } \\
  & & \ \ \ {\color{violet}\axmp \ :\ \phi\ \texttt{->}\ (\phi\ \to\ \psi)\ \texttt{->}\ \psi}
  \\
  & & {\color{violet} \} }, \\
  & &
  {\color{blue}
    (\lambda\ (\texttt{mp2.1}\ :\ (\phi\ \to\ \chi)\ (\texttt{mp2.3}\ :\ (\chi\ \to\ \psi))\ (\axmp\ \texttt{mp2.3}\ \texttt{mp2.1}))
    },
  \\
  & &
  {\color{red}
    \phi\ \texttt{->}\ (\phi\ \to\ (\psi\ \to\ \chi))\ \texttt{->}\ (\psi\ \to\ \chi) }
  \\
  & ) & = \ \False
  \end{array}$
\end{frame}

\begin{frame}
  \frametitle{$\Theta$, Probabilistic Logic Networks (PLN)}

  \begin{itemize}
  \item<+-> How likely is there a \emph{proof} of ${\color{red}\Tau}$ in
    ${\color{violet}\Gamma}$:
    $$\exists {\color{blue}\pi}\ \Theta({\color{violet}\Gamma}, {\color{blue}\pi}, {\color{red}\Tau})\ \measeq\ \$\texttt{TV}$$
  \item<+-> How likely is ${\color{blue}\Pi}$ proving a \emph{theorem} in
    ${\color{violet}\Gamma}$:
    $$\exists {\color{red}\tau}\ \Theta({\color{violet}\Gamma}, {\color{blue}\Pi}, {\color{red}\tau})\ \measeq\ \$\texttt{TV}$$
  \item<+-> How likely is there a \emph{theory} in which ${\color{blue}\Pi}$ proves
    ${\color{violet}\Tau}$:
    $$\exists {\color{violet}\gamma}\ \Theta({\color{violet}\gamma}, {\color{blue}\Pi}, {\color{red}\Tau})\ \measeq\ \$\texttt{TV}$$
  \item<+-> How likely is there a \emph{proof} of a \emph{theorem} in a
    \emph{theory} with certain \emph{properties}:
    $$\exists {\color{violet}\gamma}, {\color{blue}\pi},
         {\color{red}\tau}\ \Theta({\color{violet}\gamma},
         {\color{blue}\pi}, {\color{red}\tau}) \land
         P({\color{violet}\gamma}) \land Q({\color{blue}\pi}) \land
         R({\color{red}\tau}) \land S({\color{violet}\gamma},
         {\color{blue}\pi}, {\color{red}\tau})
         \ \measeq\ \$\texttt{TV}$$
  \end{itemize}
\end{frame}

\begin{frame}
  \frametitle{PLN Recall}
  \begin{itemize}
  \item \includegraphics[scale=0.05]{figs/NAL.jpg} Non-Axiomatic Logic (NAL), \emph{Pei Wang, 2013}
  \item \includegraphics[scale=0.02]{figs/PLN.jpg} Probabilistic Logic Networks (PLN), \emph{Ben Goertzel et al, 2008}
  \item \includegraphics[scale=0.02]{figs/SL.jpg} Subjective Logic, \emph{Audun Jøsang, 2016}\\[1cm]
  \end{itemize}
  %% \begin{center}
  %% \digraph[scale=0.5]{fromtheorytotheorem}{
  %%     rankdir=LR;
  %%     node[shape=none, label=""] A;
  %%     node[shape=none, label=""] B;
  %%     A -> B [label=""];
  %% }
  %% \end{center}
  $$\rightarrow$$
  $$\Gamma\vdash\Tau$$
  $$\leftdasharrow$$
  %% \begin{center}
  %% \digraph[scale=0.5]{fromtheoremtotheory}{
  %%     rankdir=RL;
  %%     node[shape=none, label=""] A;
  %%     node[shape=none, label=""] B;
  %%     B -> A [style=dashed, label=""];
  %% }
  %% \end{center}
\end{frame}

\begin{frame}
  \frametitle{PLN Recall}
  \begin{columns}
    \column{5cm}
    \emph{\underline{Deduction:}}\\
    \digraph[scale=0.5]{deduction}{
      rankdir=LR;
      node[shape=none, label="A"] A;
      node[shape=none, label="B"] B;
      node[shape=none, label="C"] C;
      A -> B [label=""];
      B -> C [label=""];
      A -> C [style=dashed, label=""];
    }
    \begin{center}
      \begin{prooftree}
        \AxiomC{$B \limp C$}
        \AxiomC{$A \limp B$}
        \BinaryInfC{$A \limp C$}
      \end{prooftree}
    \end{center}
    \column{5cm}
    \pause
    \emph{\underline{Induction:}}\\
    \digraph[scale=0.5]{induction}{
      rankdir=LR;
      node[shape=none, label="A"] A;
      node[shape=none, label="B"] B;
      node[shape=none, label="C"] C;
      A -> B [label=""];
      B -> C [style=dashed, label=""];
      A -> C [label=""];
    }
    \begin{center}
      \begin{prooftree}
        \AxiomC{$A \limp C$}
        \AxiomC{$A \limp B$}
        \BinaryInfC{$B \limp C$}
      \end{prooftree}
    \end{center}
    \column{5cm}
    \pause
    \emph{\underline{Abduction:}}\\
    \digraph[scale=0.5]{abduction}{
      rankdir=LR;
      node[shape=none, label="A"] A;
      node[shape=none, label="B"] B;
      node[shape=none, label="C"] C;
      A -> B [style=dashed, label=""];
      B -> C [label=""];
      A -> C [label=""];
    }
    \begin{center}
      \begin{prooftree}
        \AxiomC{$A \limp C$}
        \AxiomC{$B \limp C$}
        \BinaryInfC{$A \limp B$}
      \end{prooftree}
    \end{center}
  \end{columns}
\end{frame}

\begin{frame}
  \frametitle{PLN Recall}
  \begin{columns}
    \column{5cm}
    \underline{\emph{Truth Value:}}
    $$A \limp B\ \measeq\ \texttt{TV}$$
    %% \begin{itemize}
  %% \item
    \begin{center}
      $\texttt{TV}$\\
      =\\
      \emph{Second Order Probability Distribution}
    \end{center}
    %% \end{itemize}
    \column{10cm}
    \only<1>{\includegraphics[scale=0.4]{figs/second_order.png}}
  \end{columns}
\end{frame}

%% The following introduces glitches with the position of the figures,
%% so we duplicate the frame instead of using only.
%%
%% \begin{frame}
%%   \frametitle{PLN Recall}
%%   \begin{columns}
%%     \column{5cm}
%%     \underline{\emph{Simple Truth Value:}}
%%     $$A \limp B\ \measeq\ \STV{s}{c}$$
%%     \begin{itemize}
%%     \item $s$ = \emph{strength}
%%     \item $c$ = \emph{confidence}
%%     \item Beta Distribution
%%     \end{itemize}
%%     \column{10cm}
%%     %% \only<1>{\includegraphics[scale=0.4]{figs/bayesian_prior.png}}
%%     %% \only<2>{\includegraphics[scale=0.4]{figs/observations_0_1.png}}
%%     %% \only<3>{\includegraphics[scale=0.4]{figs/observations_0_2.png}}
%%     %% \only<4>{\includegraphics[scale=0.4]{figs/observations_0_6.png}}
%%     %% \only<5>{\includegraphics[scale=0.4]{figs/observations_0_100.png}}
%%     %% \only<6>{\includegraphics[scale=0.4]{figs/bayesian_prior.png}}
%%     %% \only<7>{\includegraphics[scale=0.4]{figs/observations_1_0.png}}
%%     %% \only<8>{\includegraphics[scale=0.4]{figs/observations_2_0.png}}
%%     %% \only<9>{\includegraphics[scale=0.4]{figs/observations_6_0.png}}
%%     %% \only<10>{\includegraphics[scale=0.4]{figs/observations_100_0.png}}
%%     %% \only<11>{\includegraphics[scale=0.4]{figs/bayesian_prior.png}}
%%     %% \only<12>{\includegraphics[scale=0.4]{figs/observations_1_1.png}}
%%     %% \only<13>{\includegraphics[scale=0.4]{figs/observations_2_4.png}}
%%     %% \only<14>{\includegraphics[scale=0.4]{figs/observations_20_40.png}}
%%   \end{columns}
%% \end{frame}

\begin{frame}
  \frametitle{PLN Recall}
  \begin{columns}
    \column{5cm}
    \underline{\emph{Simple Truth Value:}}
    $$A \limp B\ \measeq\ \STV{s}{c}$$
    \begin{itemize}
    \item $s$ = \emph{strength}
    \item $c$ = \emph{confidence}
    \item Beta Distribution
    \end{itemize}
    \column{10cm}
    \includegraphics[scale=0.4]{figs/bayesian_prior.png}
  \end{columns}
\end{frame}

\begin{frame}
  \frametitle{PLN Recall}
  \begin{columns}
    \column{5cm}
    \underline{\emph{Simple Truth Value:}}
    $$A \limp B\ \measeq\ \STV{s}{c}$$
    \begin{itemize}
    \item $s$ = \emph{strength}
    \item $c$ = \emph{confidence}
    \item Beta Distribution
    \end{itemize}
    \column{10cm}
    \includegraphics[scale=0.4]{figs/observations_0_1.png}
  \end{columns}
\end{frame}

\begin{frame}
  \frametitle{PLN Recall}
  \begin{columns}
    \column{5cm}
    \underline{\emph{Simple Truth Value:}}
    $$A \limp B\ \measeq\ \STV{s}{c}$$
    \begin{itemize}
    \item $s$ = \emph{strength}
    \item $c$ = \emph{confidence}
    \item Beta Distribution
    \end{itemize}
    \column{10cm}
    \includegraphics[scale=0.4]{figs/observations_0_2.png}
  \end{columns}
\end{frame}

\begin{frame}
  \frametitle{PLN Recall}
  \begin{columns}
    \column{5cm}
    \underline{\emph{Simple Truth Value:}}
    $$A \limp B\ \measeq\ \STV{s}{c}$$
    \begin{itemize}
    \item $s$ = \emph{strength}
    \item $c$ = \emph{confidence}
    \item Beta Distribution
    \end{itemize}
    \column{10cm}
    \includegraphics[scale=0.4]{figs/observations_0_6.png}
  \end{columns}
\end{frame}

\begin{frame}
  \frametitle{PLN Recall}
  \begin{columns}
    \column{5cm}
    \underline{\emph{Simple Truth Value:}}
    $$A \limp B\ \measeq\ \STV{s}{c}$$
    \begin{itemize}
    \item $s$ = \emph{strength}
    \item $c$ = \emph{confidence}
    \item Beta Distribution
    \end{itemize}
    \column{10cm}
    \includegraphics[scale=0.4]{figs/observations_0_100.png}
  \end{columns}
\end{frame}

\begin{frame}
  \frametitle{PLN Recall}
  \begin{columns}
    \column{5cm}
    \underline{\emph{Simple Truth Value:}}
    $$A \limp B\ \measeq\ \STV{s}{c}$$
    \begin{itemize}
    \item $s$ = \emph{strength}
    \item $c$ = \emph{confidence}
    \item Beta Distribution
    \end{itemize}
    \column{10cm}
    \includegraphics[scale=0.4]{figs/bayesian_prior.png}
  \end{columns}
\end{frame}

\begin{frame}
  \frametitle{PLN Recall}
  \begin{columns}
    \column{5cm}
    \underline{\emph{Simple Truth Value:}}
    $$A \limp B\ \measeq\ \STV{s}{c}$$
    \begin{itemize}
    \item $s$ = \emph{strength}
    \item $c$ = \emph{confidence}
    \item Beta Distribution
    \end{itemize}
    \column{10cm}
    \includegraphics[scale=0.4]{figs/observations_1_0.png}
  \end{columns}
\end{frame}

\begin{frame}
  \frametitle{PLN Recall}
  \begin{columns}
    \column{5cm}
    \underline{\emph{Simple Truth Value:}}
    $$A \limp B\ \measeq\ \STV{s}{c}$$
    \begin{itemize}
    \item $s$ = \emph{strength}
    \item $c$ = \emph{confidence}
    \item Beta Distribution
    \end{itemize}
    \column{10cm}
    \includegraphics[scale=0.4]{figs/observations_2_0.png}
  \end{columns}
\end{frame}

\begin{frame}
  \frametitle{PLN Recall}
  \begin{columns}
    \column{5cm}
    \underline{\emph{Simple Truth Value:}}
    $$A \limp B\ \measeq\ \STV{s}{c}$$
    \begin{itemize}
    \item $s$ = \emph{strength}
    \item $c$ = \emph{confidence}
    \item Beta Distribution
    \end{itemize}
    \column{10cm}
    \includegraphics[scale=0.4]{figs/observations_6_0.png}
  \end{columns}
\end{frame}

\begin{frame}
  \frametitle{PLN Recall}
  \begin{columns}
    \column{5cm}
    \underline{\emph{Simple Truth Value:}}
    $$A \limp B\ \measeq\ \STV{s}{c}$$
    \begin{itemize}
    \item $s$ = \emph{strength}
    \item $c$ = \emph{confidence}
    \item Beta Distribution
    \end{itemize}
    \column{10cm}
    \includegraphics[scale=0.4]{figs/observations_100_0.png}
  \end{columns}
\end{frame}

\begin{frame}
  \frametitle{PLN Recall}
  \begin{columns}
    \column{5cm}
    \underline{\emph{Simple Truth Value:}}
    $$A \limp B\ \measeq\ \STV{s}{c}$$
    \begin{itemize}
    \item $s$ = \emph{strength}
    \item $c$ = \emph{confidence}
    \item Beta Distribution
    \end{itemize}
    \column{10cm}
    \includegraphics[scale=0.4]{figs/bayesian_prior.png}
  \end{columns}
\end{frame}

\begin{frame}
  \frametitle{PLN Recall}
  \begin{columns}
    \column{5cm}
    \underline{\emph{Simple Truth Value:}}
    $$A \limp B\ \measeq\ \STV{s}{c}$$
    \begin{itemize}
    \item $s$ = \emph{strength}
    \item $c$ = \emph{confidence}
    \item Beta Distribution
    \end{itemize}
    \column{10cm}
    \includegraphics[scale=0.4]{figs/observations_1_1.png}
  \end{columns}
\end{frame}

\begin{frame}
  \frametitle{PLN Recall}
  \begin{columns}
    \column{5cm}
    \underline{\emph{Simple Truth Value:}}
    $$A \limp B\ \measeq\ \STV{s}{c}$$
    \begin{itemize}
    \item $s$ = \emph{strength}
    \item $c$ = \emph{confidence}
    \item Beta Distribution
    \end{itemize}
    \column{10cm}
    \includegraphics[scale=0.4]{figs/observations_2_4.png}
  \end{columns}
\end{frame}

\begin{frame}
  \frametitle{PLN Recall}
  \begin{columns}
    \column{5cm}
    \underline{\emph{Simple Truth Value:}}
    $$A \limp B\ \measeq\ \STV{s}{c}$$
    \begin{itemize}
    \item $s$ = \emph{strength}
    \item $c$ = \emph{confidence}
    \item Beta Distribution
    \end{itemize}
    \column{10cm}
    \includegraphics[scale=0.4]{figs/observations_20_40.png}
  \end{columns}
\end{frame}

\begin{frame}
  \frametitle{PLN Recall}
  \begin{columns}
    \column{7cm}
    \emph{\underline{Revision:}}\\
    \digraph[scale=0.5]{revision}{
      rankdir=LR;
      node[shape=none, label="A"] A;
      node[shape=none, label="B"] B;
      A -> B [label=""];
      A -> B [style=dashed, label=""];
      A -> B [label=""];
    }
    \begin{prooftree}
      \AxiomC{}
      \RightLabel{($e$)}
      \UnaryInfC{$A \limp B$}
      \AxiomC{}
      \RightLabel{($f$)}
      \UnaryInfC{$A \limp B$}
      \AxiomC{$e \perp f$}
      %% \RightLabel{($\text{R},e,f$)}
      \TrinaryInfC{$A \limp B$}
    \end{prooftree}
    \column{7cm}
    \begin{center}
      \includegraphics[scale=0.15]{figs/observations_0_1.png}
      \includegraphics[scale=0.15]{figs/observations_0_1.png}
    \end{center}
    \begin{center}
      \includegraphics[scale=0.2]{figs/observations_0_2.png}
    \end{center}
  \end{columns}
\end{frame}

\begin{frame}
  \frametitle{PLN Recall}
  \begin{columns}
    \column{5cm}
    \emph{\underline{Induction + Revision:}}\\
    \digraph[scale=0.4]{multiinduction}{
      rankdir=LR;
      node[shape=none, label="A₁"] A1;
      node[shape=none, label="A₂"] A2;
      node[shape=none, label="A₃"] A3;
      node[shape=none, label="A₄"] A4;
      node[shape=none, label="B"] B;
      node[shape=none, label="C"] C;
      A1 -> B [label=""];
      A2 -> B [label=""];
      A3 -> B [label=""];
      A4 -> B [label=""];
      B -> C [style=dashed, label=""];
      A1 -> C [label=""];
      A2 -> C [label=""];
      A3 -> C [label=""];
      A4 -> C [label=""];
    }\\
    \column{10cm}
    \begin{center}
      \only<1>{\includegraphics[scale=0.25]{figs/bayesian_prior.png}}
      \only<2>{\includegraphics[scale=0.25]{figs/observations_0_1.png}}
      \only<3>{\includegraphics[scale=0.25]{figs/observations_0_2.png}}
      \only<4>{\includegraphics[scale=0.25]{figs/observations_0_3.png}}
      \only<5>{\includegraphics[scale=0.25]{figs/observations_0_4.png}}
    \end{center}
  \end{columns}
      {\tiny
        \begin{prooftree}
          \AxiomC{{\color<2->{blue}$A_1 \limp C$}}
          \AxiomC{{\color<2->{blue}$A_1 \limp B$}}
          \RightLabel{{\color<2->{blue}(I)}}
          \BinaryInfC{{\color<2->{blue}$B \limp C$}}
          \AxiomC{{\color<3->{blue}$A_2 \limp C$}}
          \AxiomC{{\color<3->{blue}$A_2 \limp B$}}
          \RightLabel{{\color<3->{blue}(I)}}
          \BinaryInfC{{\color<3->{blue}$B \limp C$}}
          \RightLabel{{\color<3->{blue}(R)}}
          \BinaryInfC{{\color<3->{blue}$B \limp C$}}
          \AxiomC{{\color<4->{blue}$A_3 \limp C$}}
          \AxiomC{{\color<4->{blue}$A_3 \limp B$}}
          \RightLabel{{\color<4->{blue}(I)}}
          \BinaryInfC{{\color<4->{blue}$B \limp C$}}
          \RightLabel{{\color<4->{blue}(R)}}
          \BinaryInfC{{\color<4->{blue}$B \limp C$}}
          \AxiomC{{\color<5->{blue}$A_4 \limp C$}}
          \AxiomC{{\color<5->{blue}$A_4 \limp B$}}
          \RightLabel{{\color<5->{blue}(I)}}
          \BinaryInfC{{\color<5->{blue}$B \limp C$}}
          \RightLabel{{\color<5->{blue}(R)}}
          \BinaryInfC{{\color<5->{blue}$B \limp C$}}
        \end{prooftree}
      }
\end{frame}

\begin{frame}
  \frametitle{PLN Recall}
  \emph{\underline{Induction + Revision:}}\\
       {\tiny
         \begin{prooftree}
           \AxiomC{$A_1 \limp C$}
           \AxiomC{$A_1 \limp B$}
           \RightLabel{(I$,e$)}
           \BinaryInfC{$B \limp C\ \measeq\ \STV{1}{0.5}$}
           \AxiomC{$A_2 \limp C$}
           \AxiomC{$A_2 \limp B$}
           \RightLabel{(I$,f$)}
           \BinaryInfC{$B \limp C\ \measeq\ \STV{1}{0.5}$}
           \AxiomC{$e \perp f$}
           \RightLabel{(R$,e, f$)}
           \TrinaryInfC{$B \limp C\ \measeq\ \STV{1}{0.67}$}
           \AxiomC{$A_3 \limp C$}
           \AxiomC{$A_3 \limp B$}
           \RightLabel{(I,$g$)}
           \BinaryInfC{$B \limp C\ \measeq\ \STV{1}{0.5}$}
           \AxiomC{$e,f \perp g$}
           \RightLabel{(R$,e, f, g$)}
           \TrinaryInfC{$B \limp C\ \measeq\ \STV{1}{0.75}$}
         \end{prooftree}
       }
\end{frame}

\begin{frame}
  $$\phi \vdash_T \psi\ :=\ \exists p\ \Theta(T, p, \phi)\ \limp\ \exists q\ \Theta(T, q, \psi)$$
\end{frame}

\section{Introduction}

\end{document}
