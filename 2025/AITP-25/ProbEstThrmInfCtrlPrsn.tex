\documentclass[aspectratio=169]{beamer}

\usepackage{beamerthemesplit}
\usepackage{amsmath}
\usepackage{amsfonts}
\usepackage{amssymb}
\usepackage{cancel}
\usepackage{bussproofs}
\usepackage{graphicx}

% For ⩘ and ⩗ (requires the LuaLaTeX engine)
\usepackage{unicode-math}
\setmathfont{Stix Two Math}

% For highlighting MeTTa code
\usepackage{listings}
\usepackage{color}
\definecolor{mygreen}{rgb}{0,0.6,0}
\definecolor{mymauve}{rgb}{0.58,0,0.82}
\definecolor{dgreen}{rgb}{0,0.5,0}
\definecolor{dblue}{rgb}{0,0,0.5}
\definecolor{dred}{rgb}{0.5,0,0}
\lstset{ %
  backgroundcolor=\color{white},   % choose the background color
  basicstyle=\tiny,                % size of fonts used for the code
  breaklines=true,                 % automatic line breaking only at whitespace
  captionpos=b,                    % sets the caption-position to bottom
  commentstyle=\color{mygreen},    % comment style
  escapeinside={\%*}{*)},          % if you want to add LaTeX within your code
  keywordstyle=\color{blue},       % keyword style
  stringstyle=\color{mymauve},     % string literal style
}

\makeatletter
\newcommand{\tinier}{\@setfontsize{\srcsize}{3.5pt}{3.5pt}}
\makeatother

\makeatletter
\newcommand{\tiniest}{\@setfontsize{\srcsize}{2pt}{2pt}}
\makeatother

\mode<presentation>
{
  \usetheme{AnnArbor}
  \usecolortheme{crane}
}

\usepackage[english]{babel}
%% \usepackage[latin1]{inputenc}
\usepackage{times}
\usepackage[T1]{fontenc}

\newcommand{\limp}{\Rightarrow}
\newcommand{\TheoryT}{\texttt{Theory}}
\newcommand{\ProofT}{\texttt{Proof}}
\newcommand{\PropositionT}{\texttt{Proposition}}
\newcommand{\BoolT}{\texttt{Bool}}
\newcommand{\True}{\texttt{True}}
\newcommand{\False}{\texttt{False}}

\title{Estimating the Probability of a Conjecture to be a Theorem in PLN for Inference Control}

\author{Nil Geisweiller}

\institute[SingularityNET]
{
  \begin{center}
    \includegraphics[scale=0.2]{figs/snet-logo.png}\\[1cm]
    Artificial Intelligence and Theorem Proving 2025 (AITP-25)
  \end{center}
}

\date[AITP-25]

\begin{document}

\lstset{language=Lisp}

\begin{frame}
  \maketitle
\end{frame}

\begin{frame}
  Ternary predicate relating theories, proofs and propositions:
  $$\Theta : {\color{violet}\TheoryT} \times {\color{blue}\ProofT} \times {\color{red}\PropositionT} \to \BoolT$$

  \begin{itemize}
  \item {\color{violet}$\TheoryT$}: \emph{Typing relationships} encoding axioms and
    inference rules.
    {\color{violet}$$\{\texttt{Z}:\texttt{Nat},\ \texttt{S}:\texttt{Nat} \texttt{->} \texttt{Nat}\}$$}
  \item {\color{blue}$\ProofT$}: \emph{Inhabitant} of a type.
    {\color{blue}$$\texttt{(S (S (S Z)))}$$}
  \item {\color{red}$\PropositionT$}: \emph{Type}.
    {\color{red}$$\texttt{Nat}$$}
  \end{itemize}
  $$\Theta({\color{violet}\{\texttt{Z}:\texttt{Nat},\ \texttt{S}:\texttt{Nat}
  \texttt{->} \texttt{Nat}\}}, {\color{blue}\texttt{(S (S (S Z)))}}, {\color{red}\texttt{Nat}})
  = \True$$
\end{frame}

\begin{frame}
  Example (Propositional Calculus):
  NEXT
\end{frame}

\begin{frame}
  $$\phi \vdash_T \psi\ :=\ \exists p\ \Theta(T, p, \phi)\ \limp\ \exists q\ \Theta(T, q, \psi)$$
\end{frame}

\section{Introduction}

\end{document}
