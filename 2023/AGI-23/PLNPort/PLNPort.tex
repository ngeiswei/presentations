\documentclass[aspectratio=169]{beamer}

\usepackage{beamerthemesplit}
\usepackage{amsmath}
\usepackage{amsfonts}
\usepackage{amssymb}
\usepackage{cancel}
\usepackage{bussproofs}

% For ⩘ and ⩗ (requires the LuaLaTeX engine)
\usepackage{unicode-math}
\setmathfont{Stix Two Math}

% For highlighting MeTTa code
\usepackage{listings}
\usepackage{color}
\definecolor{mygreen}{rgb}{0,0.6,0}
\definecolor{mygray}{rgb}{0.5,0.5,0.5}
\definecolor{mymauve}{rgb}{0.58,0,0.82}
\lstset{ %
  backgroundcolor=\color{white},   % choose the background color
  basicstyle=\tiny,                % size of fonts used for the code
  breaklines=true,                 % automatic line breaking only at whitespace
  captionpos=b,                    % sets the caption-position to bottom
  commentstyle=\color{mygreen},    % comment style
  escapeinside={\%*}{*)},          % if you want to add LaTeX within your code
  keywordstyle=\color{blue},       % keyword style
  stringstyle=\color{mymauve},     % string literal style
}

% Commands for Atomese code
\newcommand{\SP}{\;\;\;}
\newcommand{\TTrue}{\textit{True}}
\newcommand{\TFalse}{\textit{False}}
\newcommand{\TAtom}{\textit{Atom}}
\newcommand{\TTime}{\textit{Time}}
\newcommand{\TEval}{\textit{Evaluation}}
\newcommand{\TList}{\textit{List}}
\newcommand{\TLamb}{\textit{Lambda}}
\newcommand{\TExec}{\textit{Execution}}
\newcommand{\TAtTime}{\textit{AtTime}}
\newcommand{\TAnd}{\textit{And}}
\newcommand{\TOr}{\textit{Or}}
\newcommand{\TNot}{\textit{Not}}
\newcommand{\TImpl}{\textit{Implication}}
\newcommand{\TPredImpl}{\textit{PredictiveImplication}}
\newcommand{\TSeqAnd}{\textit{SequentialAnd}}
\newcommand{\TSeqOr}{\textit{SequentialOr}}
\newcommand{\TBSeqAnd}{\textit{BackSequentialAnd}}
\newcommand{\TFSeqAnd}{\textit{ForeSequentialAnd}}
\newcommand{\TLag}{\textit{Lag}}
\newcommand{\TLead}{\textit{Lead}}
\newcommand{\TTV}{\textit{TV}}
\newcommand{\TTVo}{\textit{TV}_1}
\newcommand{\TTVi}{\textit{TV}_i}
\newcommand{\TTVn}{\textit{TV}_n}
\newcommand{\TTVPo}{\textit{TV}_1^P}
\newcommand{\TTVQo}{\textit{TV}_1^Q}
\newcommand{\TTVPi}{\textit{TV}_i^P}
\newcommand{\TTVQi}{\textit{TV}_i^Q}
\newcommand{\TTVPn}{\textit{TV}_n^P}
\newcommand{\TTVQn}{\textit{TV}_n^Q}
\newcommand{\TTVP}{\textit{TV}^P}
\newcommand{\TTVQ}{\textit{TV}^Q}
\newcommand{\TTVR}{\textit{TV}^R}
\newcommand{\TTVPQ}{\textit{TV}^{PQ}}
\newcommand{\TTVQR}{\textit{TV}^{QR}}
\newcommand{\TBTV}{\langle \TTV \rangle}
\newcommand{\TBTVPo}{\langle \TTVPo \rangle}
\newcommand{\TBTVQo}{\langle \TTVQo \rangle}
\newcommand{\TBTVPi}{\langle \TTVPi \rangle}
\newcommand{\TBTVQn}{\langle \TTVQn \rangle}
\newcommand{\TBTVPn}{\langle \TTVPn \rangle}
\newcommand{\TBTVQi}{\langle \TTVQi \rangle}
\newcommand{\TBTVP}{\langle \TTVP \rangle}
\newcommand{\TBTVQ}{\langle \TTVQ \rangle}
\newcommand{\TBTVR}{\langle \TTVR \rangle}
\newcommand{\TBTVPQ}{\langle \TTVPQ \rangle}
\newcommand{\TBTVQR}{\langle \TTVQR \rangle}
\newcommand{\Tstrength}{\textit s}
\newcommand{\Tconf}{\textit c}

% Commands for symbolic mathematical notations
\newcommand{\prob}[1]{\mathcal{Pr}\left(#1\right)}
\newcommand{\mean}{\textit{mean}}
\newcommand{\cnt}{\textit{count}}
\newcommand{\poscnt}{\textit{pos\_count}}
\newcommand{\sat}[1]{\mathcal{S}(#1)}
\newcommand{\ltv}[1]{<\!\!#1\!\!>}
\newcommand{\letv}[2]{(#1, #2)}
\newcommand{\limp}{\rightarrow}
\newcommand{\lpreimp}[1]{\leadsto^{#1}}
\newcommand{\lseqor}[1]{\bigslopedvee^{#1}}
\newcommand{\lseqand}[1]{\bigslopedwedge^{#1}}
\newcommand{\lbseqor}[1]{\reflectbox{$\bigslopedvee$}^{#1}}
\newcommand{\lbseqand}[1]{\reflectbox{$\bigslopedwedge$}^{#1}}
\newcommand{\ldo}[1]{\widehat{#1}}
\newcommand{\llag}[2]{\overrightarrow{#1}^{#2}}
\newcommand{\llead}[2]{\overleftarrow{#1}^{#2}}

\makeatletter
\newcommand*{\dashdownarrow}{%
  \mathrel{%
    \mathpalette\dasharrow@vert{-90}%
  }%
}
\newcommand*{\dashuparrow}{%
  \mathrel{%
    \mathpalette\dasharrow@vert{90}%
  }%
}
\newcommand*{\dasharrow@vert}[2]{%
  \sbox0{$#1\vcenter{}$}%
  \sbox2{$#1\dashrightarrow\m@th$}%
  \dimen@=1.2\dimexpr\ht2-\ht0\relax
  % 1/2 width of the new symbol with side bearing
  \sbox2{\raisebox{-\ht0}{\unhcopy2}}%
  \ht2=\z@
  \dp2=\z@
  \vcenter{\hbox to 2\dimen@{\hfill\rotatebox{#2}{\box2}\hfill}}%
}
\makeatother

\makeatletter
\newcommand{\reallytiny}{\@setfontsize{\srcsize}{2pt}{2pt}}
\makeatother

\mode<presentation>
{
  \usetheme{AnnArbor}
  \usecolortheme{crane}
}

\usepackage[english]{babel}
% \usepackage[latin1]{inputenc}
\usepackage{times}
\usepackage[T1]{fontenc}

\title{Forward, Backward, Inward, Outward and Omniward Chaining}

\author{Nil Geisweiller}

\institute[SingularityNET Foundation]
{
  \begin{center}
    SingularityNET Foundation
  \end{center}
}

\date[AGI-23: Hyperon Workshop]

\begin{document}

\begin{frame}
  \maketitle
\end{frame}

\begin{frame}
  \frametitle{Inference Tree}

  \begin{columns}
    \column{6cm}

    \begin{itemize}
    \item \alert{Formal proof} as \textit{tree}
    \item \alert{Axioms} as \textit{leaves}
    \item \alert{Theorem} as \textit{root}
    \end{itemize}

    \column{8cm}

    \begin{prooftree}
      \AxiomC{}
      \RightLabel{(P)}
      \UnaryInfC{$P$}
      \AxiomC{}
      \RightLabel{(PQ)}
      \UnaryInfC{$P \limp Q$}
      \AxiomC{}
      \RightLabel{(QR)}
      \UnaryInfC{$Q \limp R$}
      \RightLabel{(Deduction)}
      \BinaryInfC{$P \limp R$}
      \RightLabel{(Modus Ponens)}
      \BinaryInfC{$R$}
    \end{prooftree}

  \end{columns}

\end{frame}

\begin{frame}
  \frametitle{Forward Chaining}

  \begin{columns}
    \column{6cm}

    \begin{center}
      \alert{Premises\\
        $\dashdownarrow$}\\
        Conclusions
    \end{center}

    \column{8cm}

    \underline{Premises}: $P$, $P\rightarrow Q$, $Q\rightarrow R$\\

    \visible<2-3>{
    \only<-3> {
    \begin{prooftree}
      \AxiomC{}
      \RightLabel{(PQ)}
      \UnaryInfC{$P \limp Q$}
      \AxiomC{}
      \RightLabel{(QR)}
      \UnaryInfC{$Q \limp R$}
      \RightLabel{\visible<3>{(Deduction)}}
      \BinaryInfC{\visible<3>{$P \limp R$}}
    \end{prooftree}}}

    \only<4> {
    \begin{prooftree}
      \AxiomC{}
      \RightLabel{(P)}
      \UnaryInfC{$P$}
      \AxiomC{}
      \RightLabel{(PQ)}
      \UnaryInfC{$P \limp Q$}
      \AxiomC{}
      \RightLabel{(QR)}
      \UnaryInfC{$Q \limp R$}
      \RightLabel{(Deduction)}
      \BinaryInfC{$P \limp R$}
      \RightLabel{(Modus Ponens)}
      \BinaryInfC{$R$}
      \UnaryInfC{$\vdots$}
    \end{prooftree}}

  \end{columns}
\end{frame}

\begin{frame}
  \frametitle{Backward Chaining}

  \begin{columns}
    \column{6cm}

    \begin{center}
      Premises\\
      \alert{$\dashuparrow$\\
      Conclusions}
    \end{center}

    \column{8cm}

    \underline{Conclusion}: $R$

    \visible<2->{
    \begin{prooftree}
      \AxiomC{}
      \RightLabel{\visible<3-4>{(P)}}
      \UnaryInfC{\visible<3-4>{$P$}}
      \AxiomC{}
      \RightLabel{\visible<4>{(PQ)}}
      \UnaryInfC{\visible<4>{$P \limp Q$}}
      \AxiomC{}
      \RightLabel{\visible<4>{(QR)}}
      \UnaryInfC{\visible<4>{$Q \limp R$}}
      \RightLabel{\visible<4>{(Deduction)}}
      \BinaryInfC{\visible<3-4>{$P \limp R$}}
      \RightLabel{\visible<3-4>{(Modus Ponens)}}
      \BinaryInfC{$R$}
    \end{prooftree}}

  \end{columns}

\end{frame}

\begin{frame}
  \frametitle{Inward Chaining}

  \begin{columns}
    \column{6cm}

    \begin{center}
      \alert{Premises\\
        $\dashdownarrow$}\\
      $\dots$\\
      \alert{$\dashuparrow$\\
        Conclusions}
    \end{center}

    \column{8cm}

    \underline{Premises}: $P$, $P\rightarrow Q$, $Q\rightarrow R$,
    \underline{Conclusion}: $R$\\

    \visible<2->{
    \begin{prooftree}
      \AxiomC{}
      \RightLabel{(P)}
      \UnaryInfC{$P$}
      \AxiomC{}
      \RightLabel{(PQ)}
      \UnaryInfC{$P \limp Q$}
      \AxiomC{}
      \RightLabel{(QR)}
      \UnaryInfC{$Q \limp R$}
      \RightLabel{\visible<3>{(Deduction)}}
      \BinaryInfC{\visible<3>{$P \limp R$}}
      \RightLabel{(Modus Ponens)}
      \BinaryInfC{$R$}
    \end{prooftree}}

  \end{columns}

\end{frame}

\begin{frame}
  \frametitle{Outward Chaining}

  \begin{columns}
    \column{6cm}

    \begin{center}
      Premises\\
      \alert{$\dashuparrow$\\
        $\dots$\\
        $\dashdownarrow$}\\
      Conclusions
    \end{center}

    \column{8cm}

    \underline{Premise}: $P$, \underline{Lemma}: $P\rightarrow R$\\

    \visible<2->{
    \begin{prooftree}
      \AxiomC{}
      \RightLabel{(P)}
      \UnaryInfC{$P$}
      \AxiomC{}
      \RightLabel{\visible<3->{(PQ)}}
      \UnaryInfC{\visible<3->{$P \limp Q$}}
      \AxiomC{}
      \RightLabel{\visible<3->{(QR)}}
      \UnaryInfC{\visible<3->{$Q \limp R$}}
      \RightLabel{\visible<3->{(Deduction)}}
      \BinaryInfC{$P \limp R$}
      \RightLabel{\visible<4->{(Modus Ponens)}}
      \BinaryInfC{\visible<4->{$R$}}
    \end{prooftree}}

  \end{columns}

\end{frame}

\begin{frame}
  \frametitle{Omniward Chaining}

  \begin{columns}
    \column{6cm}

    \begin{center}
      Premises\\
      \alert{$\dashuparrow\dashdownarrow\dashuparrow\dashdownarrow$\\
        $\dots$\\
        $\dashdownarrow\dashuparrow\dashdownarrow\dashuparrow$}\\
      Conclusions
    \end{center}

    \column{8cm}

    \underline{Premise}: $Q\rightarrow R$, \underline{Lemma}: $P\rightarrow R$\\

    \visible<2->{
    \begin{prooftree}
      \AxiomC{}
      \RightLabel{\visible<5->{(P)}}
      \UnaryInfC{\visible<5->{$P$}}
      \AxiomC{}
      \RightLabel{\visible<3->{(PQ)}}
      \UnaryInfC{\visible<3->{$P \limp Q$}}
      \AxiomC{}
      \RightLabel{(QR)}
      \UnaryInfC{$Q \limp R$}
      \RightLabel{\visible<3->{(Deduction)}}
      \BinaryInfC{$P \limp R$}
      \RightLabel{\visible<4->{(Modus Ponens)}}
      \BinaryInfC{\visible<4->{$R$}}
    \end{prooftree}}

  \end{columns}

\end{frame}

\end{document}
