\documentclass{beamer}

\usepackage{beamerthemesplit}
\usepackage{amsmath}
\usepackage{amsfonts}
\usepackage{amssymb}
\usepackage{qtree}
\usepackage{cancel}
\usepackage{tkz-graph}
%\usepackage[pdftex]{graphicx}

\makeatletter
\newcommand{\reallytiny}{\@setfontsize{\srcsize}{2pt}{2pt}}
\makeatother

\mode<presentation>
{
  \usetheme{metropolis}
  % or ...

  %\setbeamercovered{transparent}
  % or whatever (possibly just delete it)
}


\usepackage[english]{babel}
% or whatever

\usepackage[latin1]{inputenc}
% or whatever

\usepackage{times}
\usepackage[T1]{fontenc}

\title{AS-MOSES}

\author{Nil Geisweiller}

\institute[OpenCog Foundation] % (optional, but mostly needed)
{
  OpenCog Foundation
}

\date[SingularityNetathon 2018] % (optional, should be abbreviation of conference name)
{SingularityNetathon 2018}


\AtBeginSection[]
{
  \begin{frame}<beamer>{Outline}
    \tableofcontents[currentsection,currentsection]
  \end{frame}
}

\AtBeginSubsection[]
{
  \begin{frame}<beamer>{Outline}
    \tableofcontents[currentsection,currentsubsection]
  \end{frame}
}

%\newcommand{\AND}{\textit{AND}}
%\newcommand{\OR}{\textit{OR}}
%\newcommand{\NOT}{\textit{NOT}}
\newcommand{\AND}{\land}
\newcommand{\OR}{\lor}
\newcommand{\NOT}{\lnot}

\begin{document}

\frame
{
  \maketitle
}
%% \section[Outline]{}
%% \frame{\tableofcontents}

%% \section{Introduction}

\begin{frame}[fragile]
  \frametitle{MOSES: Status}

  MOSES is a program learner
  
  \begin{itemize}
  \item MOSES: Meta-Optimization Semantic Evolutionary Search (Moshe
    Looks)
  \item C++ version
  \item Learn Combo programs
  \item To be ported for the AtomSpace
  \end{itemize}
\end{frame}

\begin{frame}[fragile]
  \frametitle{MOSES: Recall}

  What makes MOSES special
  \begin{itemize}
  \item<+-> Reduction in normal form\\
    $$f(x) = 1*x + 0\ \ \ \ \Rightarrow\ \ \ \ \ f(x) = x$$
    \begin{itemize}
    \item Avoid over-representation
    \item Increase syntax vs semantics correlation
    \item Simplify subsequent evolution
    \end{itemize}
  \item<+-> Deme management
    \begin{itemize}
    \item Islands of diverse program subspaces
    \item ``Clever'' representation building
    \end{itemize}
  \item<+-> Optimization
    \begin{itemize}
    \item Attempt to learn the fitness landscape
    \item In practice Stochastic Hillclimbing + Crossover
    \end{itemize}
  \end{itemize}

\end{frame}

\begin{frame}[fragile]
  \frametitle{AS-MOSES: Port to the AtomSpace}

  Why porting MOSES to the AtomSpace?\\

  \alert{Synergies between MOSES and the rest of OpenCog}\\

  \begin{itemize}
  \item Atomese fitness function
  \item Atomese candidate programs
  \item Search in the AtomSpace
  \item Integrate background knowledge
  \item Meta-learning
  \end{itemize}

\end{frame}

\begin{frame}[fragile]
  \frametitle{AS-MOSES: Atomese}
Program example:

$$f(x, y) = x + 2*y$$

Atomese:

{\small \begin{verbatim}
(Lambda
  (VariableList
    (Variable "x")
    (Variable "y")
  (Plus
    (Variable "x")
    (Times
      (Number 2)
      (Variable "y"))))
\end{verbatim}}

\end{frame}

\begin{frame}[fragile]
  \frametitle{AS-MOSES: Deme Representation}
Representation:

$$f(x) = [-1, 1]*x + [0, 0.5, 1]$$

Atomese:

{\small \begin{verbatim}
(Quote
  (Lambda
    (Variable "$x")
    (Plus
      (Times
        (Unquote (Variable "$k0"))
        (Variable "x"))
      (Unquote (Variable "$k1")))))
\end{verbatim}}

$k0$ and $k1$ are the knob variables

\end{frame}

\begin{frame}[fragile]
  \frametitle{AS-MOSES: Deme Representation}

  Generate all candidates with the following Atomese program:

\begin{columns}

  \column{2in}
  
{\tiny
\begin{verbatim}
(Put
  (VariableList
    (Variable "$k0")
    (Variable "$k1"))
  (Quote
    (Lambda
      (Variable "$x")
      (Plus
        (Times
          (Unquote (Variable "$k0"))
          (Variable "$x"))
        (Unquote (Variable "$k1")))))
  (Set  
    (List (Number -1) (Number 0))
    (List (Number -1) (Number 0.5))
    (List (Number -1) (Number 1))
    (List (Number 1) (Number 0))
    (List (Number 1) (Number 0.5))
    (List (Number 1) (Number 2))))
\end{verbatim}}

  \column{0.1in}
  $$\Rightarrow$$

  \column{1in}

{\Tiny
\begin{verbatim}
    (Lambda
      (Variable "$x")
      (Plus
        (Times
          (Number -1)
          (Variable "$x"))
        (Number 0)))

    (Lambda
      (Variable "$x")
      (Plus
        (Times
          (Number -1)
          (Variable "$x"))
        (Number 0.5)))

    (Lambda
      (Variable "$x")
      (Plus
        (Times
          (Number -1)
          (Variable "$x"))
        (Number 1)))
\end{verbatim}}

  \column{1in}

{\Tiny
\begin{verbatim}
    (Lambda
      (Variable "$x")
      (Plus
        (Times
          (Number 1)
          (Variable "$x"))
        (Number 0)))

    (Lambda
      (Variable "$x")
      (Plus
        (Times
          (Number 1)
          (Variable "$x"))
        (Number 0.5)))

    (Lambda
      (Variable "$x")
      (Plus
        (Times
          (Number 1)
          (Variable "$x"))
        (Number 1)))
\end{verbatim}}

\end{columns}

\end{frame}

\begin{frame}[fragile]
  \frametitle{AS-MOSES: Reduction}

\alert{Axiomatize Atomese Reduction}\\[0.5cm]

For instance
\begin{columns}

\column{2in}
  
{\tiny
\begin{verbatim}
(Evaluation (stv 1 1)
  (Predicate "reduce-to")
  (List
    (Lambda
      (Variable "$x")
      (Plus
        (Times
          (Number 1)
          (Variable "$x"))
        (Number 0)))
    (Lambda
      (Variable "$x")
      (Variable "$x"))
\end{verbatim}}

\column{0.5in}
  
{\tiny means}

\column{2in}

$$f(x) = 1*x + 0$$

\begin{center}{\tiny reduces to}\end{center}

$$f(x) = x$$

\end{columns}

\end{frame}

\begin{frame}
  \frametitle{AS-MOSES: Optimization}

  \begin{itemize}
  \item<+-> Goal: Incorporate \alert{background knowledge} (including
    meta-learning) in the optimization process
  \item<+-> Suggested solution: See optimization as \alert{reasoning
      process}. Possibly dedicated policies for efficiency.
    \begin{itemize}
    \item EDAs: Probably trivial
    \item Hillclimbing: Less trivial but still
    \end{itemize}
  \end{itemize}
\end{frame}

\begin{frame}[fragile]
  \frametitle{AS-MOSES: Optimization, Hillclimbing Example}

{\tiny Hillclimbing Axioms:
\begin{enumerate}
  \item Candidates with \alert{similar knob settings} tend to be
    \alert{syntactically similar}
\begin{verbatim}
  Implication (stv 0.8 0.2)
    Predicate "similar-knob-settings"
    Predicate "similar-syntax"
\end{verbatim}
  \item \alert{Syntactically similar} candidates tend to be loosely
    \alert{semantically similar}
\begin{verbatim}
  Implication (stv 0.4 0.01)
    Predicate "similar-syntax"
    Predicate "similar-semantics"
\end{verbatim}
  \item \alert{Semantically similar} candidates tend to have
    \alert{similar fitnesses}
\begin{verbatim}
  Implication (stv 0.6 0.1)
    Predicate "similar-semantics"
    Predicate "similar-fitness"
\end{verbatim}
  \item If candidate P1 and P2 have \alert{similar fitnesses},
    \alert{P2's fitness is close to P1's fitness}
\begin{verbatim}
  ImplicationScope (stv 1 1)
    $P1, $P2
    And
      Evaluation
        Predicate "similar-fitness"
        List
          $P1
          $P2
      Evaluation
        Predicate "fitness"
        $P1
    Evaluation
      Predicate "fitness"
      $P2
\end{verbatim}
  \end{enumerate}}

\end{frame}

\begin{frame}[fragile]
  \frametitle{AS-MOSES: Optimization}
{\small
  URE query:

\begin{verbatim}
Evaluation
  Predicate "MOSES:fitness"
  $X
\end{verbatim}

\begin{itemize}
\item URE fitness: \alert{maximize strength and confidence} of the
  query
\item Chain axioms 1 to 4, using deduction, fuzzy conjunction and
  conditional instantiation to explore neighborhood
\item Once a better candidate is found, the URE fitness will be
  incentivized to re-chain axioms 1 to 4 with that better candidate.
\item We can help the URE!
\end{itemize}
}

\end{frame}

\begin{frame}
  \frametitle{AS-MOSES: Architecture}

  Open to discussion...\\[0.5cm]

  \pause

  Personal suggestion:
  \begin{itemize}
  \item Wrap existing MOSES C++ components in Scheme/Atomese so it
    appears as if \alert{AS-MOSES is an Atomese program}
  \item Progressively infuse reasoning
  \end{itemize}

\end{frame}

\end{document}
